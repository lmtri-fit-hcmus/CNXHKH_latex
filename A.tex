\section*{BÀI LÀM}
\addcontentsline{toc}{section}{\numberline{}BÀI LÀM}
\setcounter{section}{2}
\subsection*{\textbf{\fontsize{16pt}{0pt}\selectfont A. Nhận định đúng hay sai, giải thích}}
\addcontentsline{toc}{subsection}{\numberline{}A. Nhận định đúng hay sai, giải thích}

1. \textbf{“Việt Nam quá độ lên chủ nghĩa xã hội theo đúng qui luật mà C. Mác – Engghen đã khái quát”.} là sai, Việt Nam đã bỏ qua qua CNTB (Chủ nghĩa tư bản) mà tiến thẳng đến CNXH (Chủ nghĩa xã hội) 

2. \textbf{“Tất cả mọi tôn giáo chẳng qua là sự phản ánh hư ảo vào đầu óc của con người – của những lực lượng bên ngoài chi phối cuộc sống hàng ngày của họ; chỉ là sự phản ánh trong đó những lực lượng ở trần thế đã mang hình thức những lực lượng siêu trần thế”} là sai, Vì tôn giáo vẫn có thể tiếp tục tồn tại với tư cách là một hình thức trực tiếp, nghĩa là một hình thức cảm xúc của thái độ của con người đối với lực lượng xa lạ, tự nhiên và xã hội, chừng nào con người còn chịu sự thống trị của những lực lượng đó... Do đó cơ sở thực tế của sự phản ánh có tính chất tôn giáo của hiện thực vẫn tiếp tục tồn tại và cùng với cơ sở đó thì chính ngay sự phản ánh của nó trong tôn giáo cũng tiếp tục tồn tại

3. “\textbf{Sự phát triển của sản xuất xã hội qui định hình thái qui mô và kết cấu gia đình”} là sai, mặc dù sự biến đổi quy mô gia đình Việt Nam là một tất yếu không thể tránh khỏi tuy nhiên sự thay đổi đó chỉ mang tính chất điểu chỉnh để phù hợp với tính chất xã hội, chứ không phải xã hội như thế này là gia đình phải qui chuẩn như thế kia. Quan trọng nhất là điều chỉnh nhưng vẫn giữ được những giá trị tốt đẹp của gia đình truyền thống đồng thời phát huy được điểm mạnh của gia đình hiện đại
