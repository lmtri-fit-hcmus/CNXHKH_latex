\section*{ĐỀ THI}
\addcontentsline{toc}{section}{\numberline{}ĐỀ THI}
\setcounter{section}{1}
\textbf{A. Nhận định sau đúng hay sai? Giải thích vì sao?}

1. “Việt Nam quá độ lên chủ nghĩa xã hội theo đúng qui luật mà C. Mác – Engghen đã khái quát”. (1đ)

2. “Tất cả mọi tôn giáo chẳng qua là sự phản ánh hư ảo vào đầu óc của con người – của những lực lượng bên ngoài chi phối cuộc sống hàng ngày của họ; chỉ là sự phản ánh trong đó những lực lượng ở trần thế đã mang hình thức những lực lượng siêu trần thế”. (1đ)

3. “Sự phát triển của sản xuất xã hội qui định hình thái qui mô và kết cấu gia đình”?

\textbf{B. Câu hỏi Tự Luận:}

Câu 1 (3,5đ): Phân tích quan điểm của CN Mác – Lê Nin về Giai cấp công nhân. 
Sứ mệnh lịch sử, phương hướng xây dựng Giai cấp công nhân Việt Nam.

Câu 2 (3,5đ): Sự ra đời, bản chất và chức năng của nhà nước XHCN Việt Nam. Đặc điểm, giải pháp và thực tiễn xây dựng nhà nước pháp quyền xã hội chủ nghĩa ở nước ta hiện nay.
\clearpage