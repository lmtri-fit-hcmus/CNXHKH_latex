\vspace{1cm}
\subsection*{\textbf{\fontsize{16pt}{0pt}\selectfont B. Câu hỏi tự luận}}
\addcontentsline{toc}{subsection}{\numberline{}B. Câu hỏi tự luận}
\subsubsection*{\textbf{\fontsize{15pt}{0pt}\selectfont Câu 1}}
\addcontentsline{toc}{subsubsection}{\numberline{}Câu 1}

\paragraph*{{\textbf{Quan điểm của chủ nghĩa Mác – Lê Nin về Giai cấp công nhân:}}}\mbox{}

Giai cấp công nhân là sản phẩm của cách mạng công nghiệp, ra đời và phát triển gắn liền với sự ra đời và phát triển của nền đại công nghiệp; trực tiếp hay gián tiếp vận hành các công cụ sản xuất có tính chất công nghiệp ngày càng hiện đại và xã hội hóa cao. Giai cấp công nhân là sản phẩm của nền đại công nghiệp.

Giai cấp công nhân là sản phẩm của nền công nghiệp hiện đại, lực lượng đại biểu cho sự phát triển của lực lượng sản xuất tiến bộ, cho xu hướng tiến bộ của phương thức sản xuất; là giai cấp có tinh thần triệt để cách mạnh; là giai cấp có tính tổ chức và kỉ luật cao; là giai cấp có bản chất quốc tế. Vì vậy giai cấp công nhân có sứ mệnh lịch sử toàn thế giới của giai cấp công nhân là tổ chức lãnh đạo xã hội thông qua đội tiên phong là Đảng Cộng sản để đấu tranh giải phóng giai cấp, giải phóng toàn xã hội khỏi áp bức bất công xóa bỏ CNTB xây dựng thành công chủ nghĩa cộng sản trên phạm vi toàn thế giới.

\paragraph*{{\textbf{Sứ mệnh lịch sử của Giai cấp công nhân Việt Nam}}}\mbox{}

Sứ mệnh lịch sử của giai cấp công nhân Việt Nam là xóa bỏ chế độ tư bản chủ nghĩa, xóa bỏ chế độ bóc lột, tự giải phóng, giải phóng nhân dân lao động và toàn thể nhân loại khỏi sự áp bức, bóc lột, xây dựng thành công xã hội cộng sản chủ nghĩa.

Phát triển về số lượng và chất lượng, nâng cao giác ngộ và bản lĩnh chính trị, trình độ học vấn và nghề nghiệp thực hiện “tri thức hóa công nhân”, nâng cao năng lực ứng dụng công nghệ vào sản xuất nhằm tăng năng suất, chất lượng và hiệu quả làm việc, xứng đáng với vai trò lãnh đạo cách mạng trong thời kỳ mới.

\paragraph*{{\textbf{Phương hướng xây dựng Giai cấp công nhân Việt Nam}}}\mbox{}

\textbf{Một là:} Cần định hướng lại mục tiêu của giáo dục cho sát với yêu cầu của sự nghiệp công nghiệp hóa, hiện đại hóa. Có kế hoạch đào tạo nguồn nhân lực thích ứng với mục tiêu phát triển cụ thể của từng giai đoạn.. Quan tâm đến đội ngũ giai cấp công nhân hiện nay là phải quan tâm đến trình độ văn hóa, năng lực chuyên môn, phẩm chất nghề nghiệp và ý thức chính trị của họ. Xây dựng giai cấp công nhân phải thể hiện trước hết ở việc tổ chức đào tạo bồi dưỡng nâng cao tay nghề và trình độ chuyên môn. 

\textbf{Hai là:} Phải xem công tác xây dựng Đảng, củng cố các đoàn thể quần chúng là nhiệm vụ có ý nghĩa sống còn đối với phong trào công nhân hiện nay. Thực tế cho thấy công tác xây dựng Đảng và tổ chức công đoàn, đoàn thanh niên chưa theo kịp yêu cầu phát triển của sự nghiệp đổi mới. Một mặt do áp lực của những điều kiện khách quan, mặt khác bản thân các tổ chức đảng, công đoàn cũng bộc lộ những bất cập yếu kém, tự thân không theo kịp yêu cầu của sự phát triển, nhưng không có những chấn chỉnh kịp thời.

\textbf{Ba là:} Phải thực sự chăm lo đến đời sống vật chất tinh thần của công nhân. Ký các hợp đồng lao động với công nhân phải được xem là tiêu chuẩn bắt buộc đối với các chủ doanh nghiệp. Ngoài hợp đồng lao động cần chú trọng thanh kiểm tra điều kiện làm việc và cường độ lao động, không để và không cho phép chủ lao động ép công nhân làm việc vượt quá mức về cường độ, thời gian làm việc. Sự thiếu thốn và nghèo nàn về đời sống văn hóa tinh thần sẽ làm cho đại bộ phận lao động trẻ sống và làm việc trong môi trường không có cảm hứng sáng tạo, tính tích cực xã hội không có điều kiện phát huy, lao động chắc chắn sẽ không đem lại hiệu quả mong muốn, thiệt thòi trước hết cho chính các doanh nghiệp.
\subsubsection*{\fontsize{15pt}{0pt}\selectfont\textbf{Câu 2}}
\addcontentsline{toc}{subsubsection}{\numberline{}Câu 2}

\textbf{Sự ra đời: } Ở Việt Nam, tháng Tám năm 1945, nắm vững thời cơ khi phát xít Nhật đầu hàng đồng minh, Đảng ta do Chủ tịch Hồ Chí Minh lãnh đạo đã phát động nhân dân ta vùng dậy làm cách mạng tháng Tám thành công, lập ra nước Việt Nam dân chủ cộng hòa - Nhà nước công nông đầu tiên ở Đông nam châu á.  Tháng 04/1975 đất nước hoàn toàn giải phóng, Việt Nam bước sang một giai đoạn phát triển mới - giai đoạn cả nước quá độ đi lên chủ nghĩa xã hội. Ngày nay, trước những thử thách lớn lao của thời đại, với đường lối đổi mới do Đại hội VI của Đảng đề ra và được thể chế hóa trong Hiến pháp năm 2013, Nhà nước Cộng hòa xã hội chủ nghĩa Việt Nam đã và đang từng bước đổi mới, vượt qua khủng hoảng, vững chắc đi lên theo định hướng xã hội chủ nghĩa vì mục tiêu dân giầu, nước mạnh, xã hội công bằng văn minh.

\textbf{Bản chất:} Bản chất của nhà nước cộng hòa xã hội chủ nghĩa việt nam theo Hiến pháp 2013 là nhà nước của dân, do dân và vì dân. Cụ thể:
\begin{itemize}
    \item Nhân dân là chủ thể tối cao của quyền lực nhà nước.
    \item Nhà nước Cộng hòa xã hội chủ nghĩa Việt Nam là nhà nước của tất cả các dân tộc trên lãnh thổ Việt Nam, là biểu hiện tập trung của khối đại đoàn kết toàn dân tộc.
    \item Nhà nước Cộng hòa xã hội chủ nghĩa Việt Nam được tổ chức và hoạt động trên cơ sở nguyên tắc bình đẳng trong mối quan hệ giữa nhà nước và công dân.
    \item Nhà nước Cộng hòa xã hội chủ nghĩa Việt Nam là nhà nước dân chủ và pháp quyền.
\end{itemize}
\paragraph*{{\textbf{Chức năng của nhà nước XHCN Việt Nam:}}}\mbox{}
\begin{itemize}
    \item Chức năng bảo vệ chế độ xã hội chủ nghĩa, bảo vệ an ninh chính trị, trật tự an toàn xã hội.
    \item Chức năng bảo vệ quyền tự do, dân chủ của Nhân dân.
    \item Chức năng bảo vệ trật tự pháp luật, tăng cường pháp chế xã hội chủ nghĩa.
    \item Chức năng tổ chức và quản lý kinh tế.
    \item Chức năng tổ chức và quản lý văn hóa, khoa học, giáo dục
\end{itemize}

%%%%%%%%%%%%%%%%%%%%
\paragraph*{\textbf{Đặc điểm, giải pháp và thực tiễn xây dựng nhà nước pháp quyền XHCN ở nước ta hiện nay:}}\mbox{}

\textbf{Đặc điểm:} 
\begin{itemize}
    \item Dân giàu nước mạnh, dân chủ, công bằng, văn minh.
    \item Do nhân dân làm chủ
    \item Có nền kinh tế phát triển cao dựa trên lực lượng sản xuất hiện đại và quan hệ sản xuất tiến bộ phù hợp
    \item Có nền văn hóa tiên tiến đậm đà bản sắc dân tộc
    \item Con người có cuộc sống ấm no, tư do, hạnh phúc, có điều kiện phát triển toàn diện
    \item Các dân tộc trong cộng đồng Việt Nam  bình đẳng, đoàn keetts, tôn trọng và giúp đỡ nhau cùng phát triển
    \item Có nhà nước pháp quyền xã hội chủ nghĩa của nhân dân do nhân dân, vì nhân dân do Đảng cộng sản lãnh đạo
    \item Có quan hệ hữu nghị và hợp tác với các nước trên thế giới
\end{itemize}
\hspace{1cm}\textbf{Giải pháp:}
\begin{itemize}
    \item Đẩy mạnh công nghiệp hóa, hiện đại hóa đất nước, gắn liền với phát triển kinh tế, tri thức, bảo vệ tài nguyên, môi trường
    \item Phát triển kinh tế thị trường theo hướng xã hội chủ nghĩa
    \item Xây dựng nền văn hóa tiên tiến, đậm đà bản sắc dân tộc, xây dựng con người, nâng cao đời sống nhân dân, thực hiện tiến bộ và công bằng xã hội
    \item Bảo đảm vững chắc quốc phòng và an ninh quốc gia, trật tự, an toàn xã hội
    \item Thực hiện đường lối đối ngoại độc lập, tự chủ, hòa bình, hữu nghị, hợp tác và phát triển, chủ động và tích cực hội nhập quốc tế
    \item Xây dựng nền dân chủ xã hội chủ nghĩa, thực hiện đại đoàn kết toàn dân tộc, tăng cường và mở rộng mặt trận dân tộc thống nhất
    \item Xây dựng nhà nước pháp quyền xã hội chủ nghĩa của nhân dân, do nhân dân, vì nhân dân
    \item Xây dựng Đảng trong sạch, vững mạnh
\end{itemize}
\hspace{1cm}\textbf{Thực tiễn:} 

Cơ chế kiểm soát quyền lực chưa hoàn thiện; vai trò giám sát của nhân dân chưa được phát huy mạnh mẽ. Phân công, phân cấp, phân quyền, còn thiếu vế, quan trọng là đi đôi với phân bổ ngồn lực và nâng cao chất lượng nguồn nhân lực

Dân chủ, được coi là linh hồn, sinh khí của Nhà nước pháp quyền; vấn đề pháp luật, cụ thể là câu chuyện thể chế phát triển và chất lượng công chức trong bộ máy pháp quyền, gồm cả năng lực, đạo đức, phẩm chất, uy tín, là những vấn đề cần được đảm bảo phát triển thêm trong tương lai

Cùng với đó là tính minh bạch của pháp luật và của việc thực hiện pháp luật; pháp luật phải dễ tiếp cận và được thực hiện kịp thời, thống nhất; pháp luật phải bảo vệ quyền con người, bảo vệ con người cũng đang là vấn đề hiện nay

\clearpage